\documentclass{article}


% if you need to pass options to natbib, use, e.g.:
\PassOptionsToPackage{numbers, compress}{natbib}
% before loading neurips_2025


% ready for submission
\usepackage{neurips_2025}


% to compile a preprint version, e.g., for submission to arXiv, add add the
% [preprint] option:
%     \usepackage[preprint]{neurips_2025}


% to compile a camera-ready version, add the [final] option, e.g.:
%     \usepackage[final]{neurips_2025}


% to avoid loading the natbib package, add option nonatbib:
%    \usepackage[nonatbib]{neurips_2025}


\usepackage[utf8]{inputenc} % allow utf-8 input
\usepackage[T1]{fontenc}    % use 8-bit T1 fonts
\usepackage{hyperref}       % hyperlinks
\usepackage{url}            % simple URL typesetting
\usepackage{booktabs}       % professional-quality tables
\usepackage{amsfonts}       % blackboard math symbols
\usepackage{nicefrac}       % compact symbols for 1/2, etc.
\usepackage{microtype}      % microtypography
\usepackage{xcolor}         % colors
\usepackage{amssymb,amsfonts,amsmath,mathrsfs}

\title{Notes for Flow Matching}


% The \author macro works with any number of authors. There are two commands
% used to separate the names and addresses of multiple authors: \And and \AND.
%
% Using \And between authors leaves it to LaTeX to determine where to break the
% lines. Using \AND forces a line break at that point. So, if LaTeX puts 3 of 4
% authors names on the first line, and the last on the second line, try using
% \AND instead of \And before the third author name.


\author{%
  Jingxuan Zhang\\
  Department of Computer Science and Engineering\\
  East China University of Science and Technology\\
  Shanghai, China \\
  \texttt{y21220033@mail.ecust.edu.cn} \\
  % examples of more authors
  % \And
  % Coauthor \\
  % Affiliation \\
  % Address \\
  % \texttt{email} \\
  % \AND
  % Coauthor \\
  % Affiliation \\
  % Address \\
  % \texttt{email} \\
  % \And
  % Coauthor \\
  % Affiliation \\
  % Address \\
  % \texttt{email} \\
  % \And
  % Coauthor \\
  % Affiliation \\
  % Address \\
  % \texttt{email} \\
}


\begin{document}


\maketitle

% \begin{abstract}
%   The abstract paragraph should be indented \nicefrac{1}{2}~inch (3~picas) on
%   both the left- and right-hand margins. Use 10~point type, with a vertical
%   spacing (leading) of 11~points.  The word \textbf{Abstract} must be centered,
%   bold, and in point size 12. Two line spaces precede the abstract. The abstract
%   must be limited to one paragraph.
% \end{abstract}

\section{Theory}

According to the Flow Matching paper \cite{FM_Basic}, we need to find a path between the noise distribution $p_0(x_0)$, such as Gaussian, to the data distribution $p_1(x_1)$. To go through the whole path, one possible way is to interpolate the two distributions linearly:
\begin{equation}
    x_t = (1-t)x_0 + tx_1, \quad t \in [0, 1].
\end{equation}
Here $x_0 \sim p_0(x_0)$ and $x_1 \sim p_1(x_1)$. Then we can derive the velocity field $v_t(x_t)$ as:
\begin{equation}
    v_t(x_t) = \frac{dx_t}{dt} = x_1 - x_0.
\end{equation}

In the original paper, we use $\psi_t$ to model $x$ and $u_t$ to model the derivative, such as $x_t=\psi_t(x)$ and $\frac{d}{dt}x_t = \frac{d}{dt}\psi_t(x)= u_t(\psi_t(x))$. We construct a neural network $v_t(x_t)$ to approximate the velocity field $u_t(x_t)$. The loss function is defined as:
\begin{equation}
    L = \mathbb{E}_{x_t \sim p_t\left(x_t\right)}\left[\left\|v_t\left(x_t\right)-u_t\left(x_t\right)\right\|^2\right].
\end{equation}
However, we do not know how to sample from $p_t(x_t)$ nor the ground truth $u_t(x_t)$. To solve this problem, we can represent the loss function as:
\begin{equation}
  \begin{aligned}
    L &= \mathbb{E}_{x_t \sim p_t\left(x_t\right)}\left[\left\|v_t\left(x_t\right)-u_t\left(x_t\right)\right\|^2\right] \\
    &=\mathbb{E}_{x_t \sim p_t\left(x_t\right)}\left[\left\|v_t\left(x_t\right)\right\|^2-2 \cdot v_t\left(x_t\right) \cdot u_t\left(x_t\right)+\left\|u_t\left(x_t\right)\right\|^2\right]
  \end{aligned}
\end{equation}

The middle term can be transformed using the expection notation:

\begin{equation}
  \begin{aligned}
  &\mathbb{E}_{x_t \sim p_t(x)}\left[2 \cdot v_t\left(x_t\right) \cdot u_t\left(x_t\right)\right]=2 \cdot \int v_t\left(x_t\right) \cdot u_t\left(x_t\right) \cdot p_t\left(x_t\right) \mathrm{d} x_t\\
  =&2 \cdot \int v_t\left(x_t\right) \cdot \frac{\int u_t\left(x_t \mid x_1\right) p_t\left(x_t \mid x_1\right) q\left(x_1\right) \mathrm{d} x_1}{p_t\left(x_t\right)} \cdot p_t\left(x_t\right) \mathrm{d} x_t\\
  =&2 \cdot \iint v_t\left(x_t\right) \cdot u_t\left(x_t \mid x_1\right) \cdot p_t\left(x_t \mid x_1\right) q\left(x_1\right) \mathrm{d} x_1 \mathrm{d} x_t\\
=&2 \cdot \mathbb{E}_{x_1 \sim q\left(x_1\right), x_t \sim p_t\left(x_t \mid x_1\right)}\left[v_t\left(x_t\right) \cdot u_t\left(x_t \mid x_1\right)\right]
  \end{aligned}.
\end{equation}

After we plug in the above equation, we have:
\begin{equation}
  \begin{aligned}
    L &= \mathbb{E}_{x_t \sim p_t\left(x_t\right)}\left[\left\|v_t\left(x_t\right)\right\|^2\right] - 2 \cdot \mathbb{E}_{x_1 \sim q\left(x_1\right), x_t \sim p_t\left(x_t \mid x_1\right)}\left[v_t\left(x_t\right) \cdot u_t\left(x_t \mid x_1\right)\right] \\
    &\quad + \mathbb{E}_{x_t \sim p_t\left(x_t\right)}\left[\left\|u_t\left(x_t\right)\right\|^2\right]\\
    &= \mathbb{E}_{x_t \sim p_t\left(x_t\right)}\left[\left\|v_t\left(x_t\right)\right\|^2\right] - 2 \cdot \mathbb{E}_{x_1 \sim q\left(x_1\right), x_t \sim p_t\left(x_t \mid x_1\right)}\left[v_t\left(x_t\right) \cdot u_t\left(x_t \mid x_1\right)\right] \\
    &\quad + \mathbb{E}_{x_t \sim p_t\left(x_t\right)}\left[\left\|u_t\left(x_t\right)\right\|^2\right]+\left.\left\|u_t\left(x_t \mid x_1\right)\right\|^2-\left\|u_t\left(x_t \mid x_1\right)\right\|^2\right]\\
  \end{aligned}
\end{equation}

After we swap the last two terms, we have:
\begin{equation}
  \begin{aligned}
    L &= \mathbb{E}_{x_t \sim p_t\left(x_t\right)}\left[\left\|v_t\left(x_t\right)\right\|^2\right] - 2 \cdot \mathbb{E}_{x_1 \sim q\left(x_1\right), x_t \sim p_t\left(x_t \mid x_1\right)}\left[v_t\left(x_t\right) \cdot u_t\left(x_t \mid x_1\right)\right] \\
    &\quad + \mathbb{E}_{x_t \sim p_t\left(x_t\right)}\left[\left\|u_t\left(x_t \mid x_1\right)\right\|^2\right] + \mathbb{E}_{x_t \sim p_t\left(x_t\right)}\left[\left\|u_t\left(x_t\right)\right\|^2 - \left\|u_t\left(x_t \mid x_1\right)\right\|^2\right]\\
    &= \mathbb{E}_{x_1 \sim q\left(x_1\right), x_t \sim p_t\left(x_t \mid x_1\right)}\left[\left\|v_t\left(x_t\right)\right\|^2 - 2 \cdot v_t\left(x_t\right) \cdot u_t\left(x_t \mid x_1\right) + \left\|u_t\left(x_t \mid x_1\right)\right\|^2\right] \\
    &\quad + \mathbb{E}_{x_t \sim p_t\left(x_t\right)}\left[\left\|u_t\left(x_t\right)\right\|^2 - \left\|u_t\left(x_t \mid x_1\right)\right\|^2\right]\\
    &= \mathbb{E}_{x_1 \sim q\left(x_1\right), x_t \sim p_t\left(x_t \mid x_1\right)}\left[\left\|v_t\left(x_t\right) - u_t\left(x_t \mid x_1\right)\right\|^2\right] \\
    &\quad + \mathbb{E}_{x_t \sim p_t\left(x_t\right)}\left[\left\|u_t\left(x_t\right)\right\|^2 - \left\|u_t\left(x_t \mid x_1\right)\right\|^2\right]\\
  \end{aligned}
\end{equation}

The last term is a constant that does not depend on $v_t$. Therefore, we can minimize the following loss function instead:
\begin{equation}
    L = \mathbb{E}_{x_1 \sim q\left(x_1\right), x_t \sim p_t\left(x_t \mid x_1\right)}\left[\left\|v_t\left(x_t\right) - u_t\left(x_t \mid x_1\right)\right\|^2\right].
\end{equation}

Suppose $\psi_t(x)=\sigma_t\left(x_1\right) x+\mu_t\left(x_1\right)$, we have $\psi_t\left(x_0\right)=\sigma_t\left(x_1\right) x_0+\mu_t\left(x_1\right)$. Let $\mu_t\left(x_1\right)=t x_1,\quad \sigma_t\left(x_1\right)=1-t$, then $x_t=\psi_t\left(x_0\right)=(1-t) x_0+t x_1,$ which is the same as our linear interpolation. 

Now, remember that $\frac{d}{dt}x_t = \frac{d}{dt}\psi_t(x)= u_t(\psi_t(x))$. We can add a condition to the both sides, and then we have $\frac{\mathrm{d}}{\mathrm{~d} t} \psi_t(x)=u_t\left(\psi_t(x) \mid x_1\right)$ or $\frac{\mathrm{d}}{\mathrm{~d} t} x_t=u_t\left(x_t \mid x_1\right)$ or $\frac{\mathrm{d}}{\mathrm{~d} t} \psi_t\left(x_0\right)=u_t\left(\psi_t\left(x_0\right) \mid x_1\right)$.

Therefore, the loss function can be rewritten as:
\begin{equation}
    L = \mathbb{E}_{x_1 \sim q\left(x_1\right), x_0 \sim p_0\left(x_0\right)}\left[\left\|v_t\left(\psi_t\left(x_0\right)\right) - \frac{\mathrm{d}}{\mathrm{~d} t} \psi_t\left(x_0\right)\right\|^2\right].
\end{equation}
After calculating the derivative, we have:
\begin{equation}
    L = \mathbb{E}_{x_1 \sim q\left(x_1\right), x_0 \sim p_0\left(x_0\right)}\left[\left\|v_t\left((1-t) x_0+t x _1\right) - (x_1 - x_0)\right\|^2\right].
\end{equation}
This is the final loss function used in the Flow Matching paper.

% \section*{References}
\bibliographystyle{IEEEtran}
\bibliography{IEEE_Trans}


\end{document}
